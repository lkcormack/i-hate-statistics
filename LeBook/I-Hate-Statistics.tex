% Options for packages loaded elsewhere
\PassOptionsToPackage{unicode}{hyperref}
\PassOptionsToPackage{hyphens}{url}
%
\documentclass[
]{book}
\usepackage{amsmath,amssymb}
\usepackage{lmodern}
\usepackage{iftex}
\ifPDFTeX
  \usepackage[T1]{fontenc}
  \usepackage[utf8]{inputenc}
  \usepackage{textcomp} % provide euro and other symbols
\else % if luatex or xetex
  \usepackage{unicode-math}
  \defaultfontfeatures{Scale=MatchLowercase}
  \defaultfontfeatures[\rmfamily]{Ligatures=TeX,Scale=1}
\fi
% Use upquote if available, for straight quotes in verbatim environments
\IfFileExists{upquote.sty}{\usepackage{upquote}}{}
\IfFileExists{microtype.sty}{% use microtype if available
  \usepackage[]{microtype}
  \UseMicrotypeSet[protrusion]{basicmath} % disable protrusion for tt fonts
}{}
\makeatletter
\@ifundefined{KOMAClassName}{% if non-KOMA class
  \IfFileExists{parskip.sty}{%
    \usepackage{parskip}
  }{% else
    \setlength{\parindent}{0pt}
    \setlength{\parskip}{6pt plus 2pt minus 1pt}}
}{% if KOMA class
  \KOMAoptions{parskip=half}}
\makeatother
\usepackage{xcolor}
\usepackage{longtable,booktabs,array}
\usepackage{calc} % for calculating minipage widths
% Correct order of tables after \paragraph or \subparagraph
\usepackage{etoolbox}
\makeatletter
\patchcmd\longtable{\par}{\if@noskipsec\mbox{}\fi\par}{}{}
\makeatother
% Allow footnotes in longtable head/foot
\IfFileExists{footnotehyper.sty}{\usepackage{footnotehyper}}{\usepackage{footnote}}
\makesavenoteenv{longtable}
\usepackage{graphicx}
\makeatletter
\def\maxwidth{\ifdim\Gin@nat@width>\linewidth\linewidth\else\Gin@nat@width\fi}
\def\maxheight{\ifdim\Gin@nat@height>\textheight\textheight\else\Gin@nat@height\fi}
\makeatother
% Scale images if necessary, so that they will not overflow the page
% margins by default, and it is still possible to overwrite the defaults
% using explicit options in \includegraphics[width, height, ...]{}
\setkeys{Gin}{width=\maxwidth,height=\maxheight,keepaspectratio}
% Set default figure placement to htbp
\makeatletter
\def\fps@figure{htbp}
\makeatother
\setlength{\emergencystretch}{3em} % prevent overfull lines
\providecommand{\tightlist}{%
  \setlength{\itemsep}{0pt}\setlength{\parskip}{0pt}}
\setcounter{secnumdepth}{5}
\usepackage{booktabs}
\usepackage{amsthm}
\makeatletter
\def\thm@space@setup{%
  \thm@preskip=8pt plus 2pt minus 4pt
  \thm@postskip=\thm@preskip
}
\makeatother
\ifLuaTeX
  \usepackage{selnolig}  % disable illegal ligatures
\fi
\usepackage[]{natbib}
\bibliographystyle{apalike}
\IfFileExists{bookmark.sty}{\usepackage{bookmark}}{\usepackage{hyperref}}
\IfFileExists{xurl.sty}{\usepackage{xurl}}{} % add URL line breaks if available
\urlstyle{same} % disable monospaced font for URLs
\hypersetup{
  pdftitle={I Hate Statistics (and you can too)},
  pdfauthor={Lawrence K. Cormack},
  hidelinks,
  pdfcreator={LaTeX via pandoc}}

\title{I Hate Statistics (and you can too)}
\author{Lawrence K. Cormack}
\date{2023-05-15}

\begin{document}
\maketitle

{
\setcounter{tocdepth}{1}
\tableofcontents
}
\hypertarget{preface}{%
\chapter*{Preface}\label{preface}}
\addcontentsline{toc}{chapter}{Preface}

When I was first introduced to statistics-as-practiced-in-psychology in an Experimental Methods class at the University of Florida, I didn't really hate it, I just thought it was --- well --- stupid. I grew up around scientists --- most of my father's friends were ``hard'' scientists, and I volunteered over summers in a field ion microscopy lab since the 5th grade --- and I knew that scientists did \textbf{NOT} put their data in a magic box that produced a binary output of ``significant'' or ``not significant''. Fortunately, I worked as an RA in the lab of Keith White, and he was from the tradition of plotting data and using them to evaluate theoretical predictions rather than ``testing'' data to see if they were ``significant'' or not. Keith suggested I take the two semester sequence in Mathematical Statistics rather than the normal statistics sequence out of Psychology, which I did. I rather enjoyed it but, as I recall, we mostly just proved theorems and such.

I truly learned to hate statistics during my gap year at Vanderbilt University. We did a simple experiment to test two competing theories, one predicted a positive relationship between two variables, while the other predicted no relationship. The data were very clean, and clearly showed no relationship, which pleased me because this is what I had predicted before the experiment. I showed a graduate student the results and they began saying blah blah blah you can't prove the null blah blah negative result blah blah. I said we weren't trying to ``prove the null'', but the data clearly supported one theory over the other. They said something like ``your experiment didn't work, it's a negative result, it's just that simple.'' I asked ``Then what about the Michelson-Morely experiment? That was a negative result.'' They thought about it a moment and said ``That's an exception.''

I thought that, surely, when the PI got in an saw the data, they would see it my way and be equally excited. But, nope, they echoed all the points that the graduate student had made and it just baffled me. I went to Berkeley for graduate school.

At Berkeley, first year student were required to take a course called ``biostatistics''. I asked for an exemption because of my heavy math background and it was granted. I was thankful because I \emph{really} didn't want to take a stats course that I knew I would think was stupid and a waste of time while also trying to learn a whole bunch of anatomy (in which I had never had a single class) for the course I was TAing. Because the field I was gravitating towards, low level vision science, didn't rely on ``statistics'' as such, I thought I was free from them (it?) forever.

After 5 wonderful years at Berkeley, I landed a job at The University of Texas at Austin and, as fate would have it, I was immediately tasked with teaching --- you guessed it --- statistics. The course was actually called ``Experimental Psychology'', so I began to review ``Experimental Psychology'' textbooks. I was, frankly, shocked. There were chapters on ``Scales of Measurement'' yet not a peep about fitting curves to data. There were cautions that ``correlation does not imply causation'' but no mention that no other statisical ``test'' does either. I ended up not using a textbook; I made the students buy Strunk \& White \citeyearpar{strunk} and the APA Publication Manual \citeyearpar{apaman} to cover the writing, and I covered all the data analysis using notes and handouts.

Happily, I soon discovered that I was not alone; there were like-minded souls around, such as Geoff Loftus at Washington, who gave a talk at UT not long after I got here. See \citep{loftus1996} for a nice treatment of some of the same issues I rant about here.

In the last 3 decades, as it turns out, many people have declared their hatred for statistics (though not in those terms). Special issues have been devoted to debating the way statistics is (are?) practiced, books have been written on the matter, and the American Statistical Association has come out against the traditional way of doing things.

So why this rant? It's because the ways of old still linger\ldots{}

\begin{itemize}
\tightlist
\item
  obviously, there is a huge amount literature produced using them
\item
  many (most?) researchers in psychology and other fields still use them
\item
  every student I've ever had that has either taken statistics in high school or taken it at another college or university has been indoctrinated into them
\end{itemize}

It's time for this to stop! I hate statistics, and you can too!

\hypertarget{intro}{%
\chapter{Introduction}\label{intro}}

In 1925, a very influential book was published. The book was ``Statistical Methods for Research Workers'' by Ronald A. Fisher \citeyearpar{fisher}; it was a ``cookbook'' with recipes for analyzing data. There were only certain fields in those days, such as astronomy, in which the workers in the field were also accomplished mathematicians who understood probability theory and were able to compute probability estimates themselves. In fact, ``statistics'' in the 19th and early 20th century was mainly associated with quantifying properties of the state (hence the name) such as population and birth rate.

The early and now-famous statisticians such as Pearson and Student (Gossett) published tables of pre-computed probabilities so that their theories could be more easily put into practice \citep[\citet{student1908}]{pearson1900}. Unfortunately, the average worker at an agricultural research station probably didn't spend their evenings reading journals such as ``Biometrika''; even if they tried, the shear volume of complex equations in the papers might have put many practical researchers off.

Fisher's book changed all that. It provided clear examples of data analysis for various common situations without presenting any derivations or proofs. It also provided tables for looking up approximate probabilities for common experimental situations. It was probably the first ``textbook'' on statistics as we know it today.

Unfortunately, it is human nature to treat things they don't understand that come from authority figures and that other people think are right as dogma. This, combined with some poor and seemingly offhand choices in the book lead to disastrous consequencesruined psychology for generations.

I hate it.

\hypertarget{signif}{%
\chapter{Significance}\label{signif}}

What did you think when you saw this chapter title?

\hypertarget{words}{%
\chapter{Words Matter}\label{words}}

\hypertarget{significant}{%
\section{``significant''}\label{significant}}

Almost anything else would have been better:
Intriguing. Curious. Interesting.

\hypertarget{critical}{%
\section{``critical''}\label{critical}}

Again, almost anything else would have been better:
Threshold. Limen. Cut-off. Decision boundary.

\hypertarget{error}{%
\section{``error''}\label{error}}

We throw that term around a lot, but we have to realize that it's confusing to students. Perhaps using ``noise'' in general would be better.

\hypertarget{confidence-interval}{%
\section{confidence interval}\label{confidence-interval}}

This one is interesting. Given the way in which pedantic statistics teachers use it (which is technically correct), it's just stupid and wrong.

But why not teach students what they technically mean, but also allow them to be interpreted as, you know, \emph{confidence} intervals? Here's what I mean:

\hypertarget{null}{%
\chapter{The Null Hypothesis}\label{null}}

Okay, I'm so damn sick of hearing ``you can't prove the null'' I think I'll cry the next time I read it somewhere else. It often followed by something like ``you can only fail to reject the null.'' While this is true in philosophy class, and \emph{technically} True with a capital T in statistics\ldots{}

We \textbf{can} prove the null \textbf{for all practical intents and purposes}. We just can. And we do. If we didn't, science would proceed at a much slower pace.

\hypertarget{concl}{%
\chapter{Conclusions}\label{concl}}

I hate statistics, but at least the future is bright!

  \bibliography{book.bib}

\end{document}
